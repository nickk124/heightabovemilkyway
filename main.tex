\documentclass[12pt]{article}
\usepackage[margin=1in]{geometry}
\usepackage{amsmath}
\usepackage{amssymb}
\usepackage{mathtools}
\usepackage{graphicx}
\usepackage{lmodern}
\usepackage{float}
\usepackage{physics}
\usepackage{minibox}
\usepackage{listings}
\usepackage{color}
\usepackage{comment}
\usepackage{xcolor}
\colorlet{mygray}{black!30}
\colorlet{mygreen}{green!60!blue}
\colorlet{mymauve}{red!60!blue}
\lstset{
	backgroundcolor=\color{gray!10},  
	basicstyle=\ttfamily,
	columns=fullflexible,
	breakatwhitespace=false,      
	breaklines=true,                
	captionpos=b,                    
	commentstyle=\color{mygreen}, 
	extendedchars=true,              
	frame=single,                   
	keepspaces=true,             
	keywordstyle=\color{blue},      
	language=python,                 
	numbers=none,                
	numbersep=5pt,                   
	numberstyle=\tiny\color{blue}, 
	rulecolor=\color{mygray},        
	showspaces=false,               
	showtabs=false,                 
	stepnumber=5,                  
	stringstyle=\color{mymauve},    
	tabsize=3,                      
	title=\lstname       
}

\newcommand*\diff{\mathop{}\!\mathrm{d}}
\newcommand{\normal}{\hat{\mathbf{n}}}
\newcommand{\epn}{\varepsilon_0}
\newcommand{\vv}[1]{\tilde{\mathbf{#1}}}
\newcommand{\unit}[1]{\hat{\mathbf{#1}}}
\newcommand{\epc}{\tilde{\varepsilon}}
\newcommand{\diag}{\text{diag}}
\author{Nick Konz}
\title{ERIRA Cygnus Project Math (Full 3D Version)}

\begin{document}
	\maketitle
	\section{What You Know}
	\par	Given: 
	\begin{enumerate}
		\item $r_c = 8.1$ kpc (distance from Earth to Galactic Center).
		\item $v = 220$ km/s (tangential/rotational velocity of Earth around the galaxy.
		\item Galactic longitude $\ell$ of our observations (taken from MW data)
		\item Observed angular separation $\delta$ between blue and green arms
		\item Observed wavelengths of green and blue arms: $\lambda_b$, $\lambda_g$, as well as the rest-frame hydrogen wavelength, $\lambda_0=1420.41$ MHz.
	\end{enumerate}
	\par Assumptions/Approximations:
	\begin{enumerate}
		\item Tangential velocity of the blue and green and green arm regions of interest are also $v$.
	\end{enumerate}
	\begin{figure}[H]
		\centering
		\includegraphics[width=1.0\linewidth]{"../Reichart Research/ERIRA 2019/cygnus/math"}
		\caption{Geometry of the Problem}
		\label{fig:math}
	\end{figure}
	\section{Setting Up the Problem}
	\par Observe the following diagram that illustrates the geometry of the full problem, fig. \ref{fig:math}. Note the labeled angles and distances, and that the goal is to find the height of us above the galactic plane (assuming a flat Milky Way), $h$. From here, we can first use Doppler shift to find 
	\begin{equation}
		c\frac{\lambda_g-\lambda_0}{\lambda_0}=v_{LOS, g}=v\cos\beta\cos\theta \qquad c\frac{\lambda_b-\lambda_0}{\lambda_0}=v_{LOS, b}=v\cos\xi\cos\tau.
	\end{equation}
	Also note that
	\begin{equation}
		\tan\theta=\frac{h}{d_{rg}} \qquad \tan\tau=\frac{h}{d_{rg}+d_{gb}}.
	\end{equation}
	Considering the horizontal plane, use basic trig to find
	\begin{equation}
		\alpha = \frac{\pi}{2} - \beta \qquad \eta = \frac{\pi}{2} - \xi \qquad \lambda = \pi - \alpha \qquad \vartheta=\pi-\lambda-\eta \qquad \gamma=\pi-\ell-\alpha,
	\end{equation}
	and then the Law of Sines to obtain
	\begin{equation}
		r_g=\frac{r_c}{\sin\alpha}\sin\ell \qquad r_b=\frac{r_c}{\sin\eta}\sin\lambda.
	\end{equation}
		Using the Law of Cosines, you then get that
	\begin{equation}
		d_{gb}=\sqrt{r_g^2+r_b^2-2r_br_g\cos\vartheta} \qquad d_{rg}=\sqrt{r_c^2+r_g^2-2r_cr_g\cos\gamma}.
	\end{equation}
		Next, look at the vertical plane. Observe that
	\begin{equation}
	\tan\sigma=\frac{d_{rg}}{h} \qquad \tan(\sigma+\delta)=\frac{d_{rg}+d_{gb}}{h}.
	\end{equation}
	Use the obscure trig identity of $\displaystyle \tan(\sigma+\delta)=\frac{\tan(\sigma)+\tan(\delta)}{1-\tan(\sigma)\tan(\delta)}$ with the equation on the right of (6), you can then plug in the left of (6) into that, separate things  into powers of $h$, and you'll get a quadratic equation for $h$. Solve this with the quadratic formula to obtain 
	\begin{equation}
	h=\frac{d_{gb}\pm\sqrt{d_{gb}^2-4\tan^2(\delta)(d_{rg}+d_{gb})d_{rg}}}{2\tan(\delta)}.
	\end{equation}
	\section{The Hard Part}
	\par In the way that we normally approach this, we assume that the line of sight (LOS) velocity vectors that we're observing of the arms lie within the plane of the Milky Way (a paradoxical assumption considering that the whole point of this is to find our height above the galaxy). This version is harder because we DON'T make this assumption (if we did, $\theta$ and $\tau$ in equation (1) would just be zero). As such, finding $\theta$ and $\tau$ are required, and yet from equation (2), these angles rely on $h$, the very thing that we're trying to find. This is where it gets harder.
	\par What you need to do is get equations (5) only in terms of $h, d_{rg}$ and $d_{gb}$, by starting from equation (1) and working downwards. You'll need to use the fact that
	\begin{equation}
		\beta=\arccos\left[\frac{v_{LOS,g}}{\displaystyle v\cos\left(\arctan\left[\frac{h}{d_{rg}}\right]\right)}\right] \qquad \xi=\arccos\left[\frac{v_{LOS,b}}{\displaystyle v\cos\left(\arctan\left[\frac{h}{d_{rg}+d_{gb}}\right]\right)}\right].
	\end{equation}
	From equation (7), you can then write $h$ in terms of $h, d_{rg}$ and $d_{gb}$, such that (5) are solely in terms of $d_{rg}$ and $d_{gb}$. Finally, solve equations (5) numerically; you have these two equations and two unknowns of $d_{rg}$ and $d_{gb}$, so they are solvable (I just plotted them both in $d_{rg}$ vs. $d_{gb}$ in Python). The way that I actually implemented this is by plotting 
	\begin{equation}
	\sqrt{r_g^2+r_b^2-2r_br_g\cos\vartheta} - d_{gb}
	\end{equation} and 
	\begin{equation}
	\sqrt{r_c^2+r_g^2-2r_cr_g\cos\gamma} - d_{rg},
	\end{equation}
	 and seeing where they intersected.
	Plug in these values of $d_{rg}$ and $d_{gb}$ into (7), and solve for $h$. My code (sorry for any small mistakes) is as follows:
	\section{Python 3 Implementation}
	\begin{lstlisting}
	#----- Author: Nick Konz -----#
	import argparse
	from argparse import RawTextHelpFormatter
	import numpy as np
	import astropy as ap
	import scipy as sp
	import matplotlib.pyplot as plt
	
	rC = 8.1 #kpc
	v = 220 #km/s
	l = np.radians(83)
	delta = np.radians(5)
	c = 3e5
	
	fN = 1420.41e6 #normal
	fR = 1420.5e6 #wavelengths for R, G, B
	fG = 1420.71e6
	fB = 1420.85e6
	
	def w(f):
		return (3e8)/f
	
	wN = w(fN)
	wR = w(fR)
	wG = w(fG)
	wB = w(fB)
	
	pm = 1;
	
	def wDiff(w):
		return w - wN
	
	vLOSg = c * wDiff(wG) / wN
	vLOSb = c*wDiff(wB) / wN
	
	
	def h(dRG, dGB):
		return (dRG + dGB - dRG + pm*np.sqrt(dGB**2 - 4*(dRG + dGB)*dRG*(np.tan(delta)**2))) / 2*np.tan(delta)
	
	def beta(dRG, dGB):
		return np.arccos(vLOSg/(v*np.cos(np.arctan(h(dRG, dGB)/dRG))))
	
	def alpha(dRG, dGB):
		return np.pi/2 - beta(dRG, dGB)
	
	def gamma(dRG, dGB):
		return np.pi - 1 - alpha(dRG, dGB)
	
	def xi(dRG, dGB):
		return np.arccos(vLOSb/(v*np.cos(np.arctan(h(dRG, dGB)/(dRG+dGB)))))
	
	def eta(dRG, dGB):
		return np.pi/2 - xi(dRG, dGB)
	
	def lambd(dRG, dGB):
		return np.pi - alpha(dRG, dGB)
	
	def rB(dRG, dGB):
		return rC*np.sin(lambd(dRG, dGB))/np.sin(eta(dRG, dGB))
	
	def rG(dRG, dGB):
		return rC*np.sin(l)/np.sin(alpha(dRG, dGB))
	
	def vtheta(dRG, dGB):
		return np.pi - lambd(dRG, dGB) - eta(dRG, dGB)
	
	def f(dRG, dGB): # = 0
		return np.sqrt(rC**2 + rG(dRG, dGB)**2 - 2*rC*rG(dRG, dGB)*np.cos(gamma(dRG, dGB))) - dRG
	
	def g(dRG, dGB): # = 0
		return np.sqrt(rG(dRG, dGB)**2 + rB(dRG, dGB)**2 - 2*rB(dRG, dGB)*rG(dRG, dGB)*np.cos(vtheta(dRG, dGB))) - dGB
		
	def main():
	
		x = np.linspace(-50,50,1000)
		y = np.linspace(-50,50,1000)
		X, Y = np.meshgrid(x,y)
		
		F = f(X,Y)
		G = g(X,Y)	
		Z = h(X,Y)
		
		plt.figure(1)
		plt.imshow(Z, extent=[-50, 50, -50, 50], origin='lower',
		cmap='viridis', alpha=0.8)
		plt.colorbar(label="height above Milky Way [kPC]")
		plt.axis(aspect='image');
		plt.contour(X,Y,F,[0], label="Condition 1")
		plt.contour(X,Y,G,[0], cmap=plt.cm.hot, label="Condition 2")
		plt.title("Fit for Earth's Height Above a Flat Milky Way")
		plt.ylabel('$d_{rg}$ (distance from Earth to green arm [kPC])')
		plt.xlabel('$d_{gb}$ (distance from green arm to blue arm [kPC])')
		plt.legend()
		plt.yscale('log')			
		plt.show()

	main()
	\end{lstlisting}
\end{document}